
\song{Gedeón}{(Hosana 2 - 59; Hlahol 2 - str. 74)}{

\ref \G Gede\C ón, Gede\Emiv7 ón, Gede\Ami ón, Gede\G ón, dal první \C tón, první \Emiv7 tón, první \Ami tón, první \G tón. 
      
\s \C Vezmi milý Gedeóne ze svých mužů \Emiv7 tři sta, \Ami přes přesilu protivníka je tvá výhra \G jistá. 
   \F Neboj a věř, \C neboj a \G věř. \\*
   \C Budu s tebou praví Bůh až půjdeš na Mid\Emiv7 jánce, \Ami jejich vojsko nestačí si ani sbalit \G rance.
   \F Neboj a věř, \Emi neboj a \C věř. \G

\s Jak Bůh řekl, tak se taky zanedlouho stalo, Gedeón se nestrachoval, že má mužů málo. Neboj a věř, neboj a věř.\\*
   Beze zbraně obstupují nepřátelské pole, tma se zatím potichounku ujímá své role. Neboj a věř, neboj a věř.

\s A když je tma nad krajem už dostatečně hustá, přikládá si Gedeón svou polnici na ústa. Neboj a věř, neboj a věř.\\*
  Za chviličku s Gedeónem tři sta mužů troubí, nepřátelský tábor bude brzy místem zhouby. Neboj a věř, neboj a věř.

\s V rytmu tónů poskakují pochodňová světla, pro ospalé nepřátele je to pravá metla. Neboj a věř, neboj a věř.\\*
  Midjánce to představení čím dál více mate, utíkají, skáčou, pletou páté přes deváté. Neboj a věř, neboj a věř.

\s Tři sta mužů, jak je vidět nepřítele složí, přesvědčivě dokázal to Gedeón - muž Boží. Neboj a věř, neboj a věř.\\*
  Bůh ho vedl, když vytáhl tenkrát na Midjánce, nepřátelé nestačili ani sbalit rance. Neboj a věř, neboj a věř.

}
